\documentclass[11pt]{article}
\usepackage{amsmath}
\usepackage{amssymb}
\usepackage[backend=bibtex, style=authoryear]{biblatex}


\usepackage[mmddyyyy,hhmmss]{datetime}
\usepackage{xcolor}

\definecolor{hyperblue}{rgb}{0,0,0.7}

\usepackage{hyperref}[hidelinks]

\hypersetup{
	colorlinks = true, %Colours links instead of ugly boxes
	urlcolor = hyperblue, %Colour for external hyperlinks
	linkcolor = hyperblue, %Colour of internal links
    anchorcolor = hyperblue,
    citecolor = hyperblue,
    filecolor = hyperblue,
     }

\definecolor{nicholasCol}{RGB}{203,97,63}
\definecolor{amyCol}{RGB}{255,20,147}
\newcommand{\nicholas}[1]{{\color{nicholasCol} [\textbf{NS:} #1 (\today\ \currenttime)]}}
% \newcommand{\amy}[1]{{\color{amyCol} [\textbf{Amy:} #1 (\today\ \currenttime)]}}
\newcommand{\amy}[1]{{\color{amyCol} [\textbf{Amy:} #1]}}

% \addbibresource{~/Dropbox/TAGreview/PhDreferences.bib}

\textwidth=470pt
\oddsidemargin=0pt
\topmargin=0pt
\headheight=0pt
\textheight=650pt
\headsep=0pt

\pdfpagewidth=\paperwidth
\pdfpageheight=\paperheight

\title{Contribution to Diversity Equity and Inclusion}


\author{Nicholas Santantonio}
\date{\today}


\newcommand{\gxe}{G$\times$E}
\newcommand{\gxg}{G$\times$G}


\begin{document}

% \amy{Amy is gonna review my DEI statement! Yay!}

\section*{\centering Statement of Diversity, Equity and Inclusion}
\begin{center} Nicholas Santantonio \end{center}
% \subsection*{Introduction}

% Need to say something about me/what I have done in the past. High school dropout? 

% high school dropout, dead end jobs, . know what it feels like to be lost.
\noindent \fbox{
\begin{minipage}{0.95 \linewidth}
	\emph{This is a long version of my statement of contribution to diversity, equity and inclusion. I fully intend to pursue all of the ideas presented below, regardless of where I find academic employment.}
	% I have been thinking about the ideas presented below are initiatives I have engaged in or intend to once I am hired, and are not simply .
\end{minipage}
}

\medskip
% Can high potential be found by looking at grade variability. We use the mean (GPA), but this statistic only describes the center of the distribtion, not its breadth or shape. Hypothesis: students with highly variable grades are more likely to suceed than students with less variable grades given they have the same GPA. 



Everyone deserves an equal chance to show their potential. Often, it takes the right time, place, person or simply an opportunity to inspire someone to set out on a path that will lead them to success. It took several tries for me. Two years of working dead-end jobs as a high school dropout was enough to encourage me to pursue an education at New Mexico State University. But even there I struggled, failing to meet a 2.0 GPA by the end of my fourth semester. With the help of several professors at NMSU and continued support from my family, I found a passion for genetics during my third year and everything changed.

I am one of the privileged ones. Without that support system, I may not have made it far enough to find that spark. So I ask myself, how can I provide adequate support and opportunity to help create that spark in those far less privileged than I? As an educator, I can work to create opportunities for students from underrepresented groups, including women, to pursue further education and expose students to different cultural perspectives in and outside of the classroom. I can take active measures to include students as part of a healthy lab culture, where expectations are clear and everyone is given equal opportunity to succeed in their own pursuits. Importantly, I can continue my own education into diversity, equity and inclusion (DEI) so that I can adapt my own efforts to better serve the community. %\amy{lot's of I cans}\nicholas{yes, but thats the point of answering "how can I". }
% NMSU is a minority serving institution. Mentored high school student during my Masters.

% On more than one occasion I have found myself using my own experience with being lost to explain to a concerned parent that all is not indeed lost. I encouraged them to continue to support their child because you never know when that spark might occur. 

\subsection*{The Diversity Preview Weekend}

To engage with the community and apply my skills to help the DEI effort, I joined the Diversity Preview Weekend (\href{https://cornelldpw.org}{DPW}) initiative at Cornell University as a co-leader in September 2019, currently serving as the fundraising chair. DPW aims to increase DEI in STEM graduate programs at Cornell and around the United States by inviting applicants from underrepresented groups to attend a fully-funded weekend at Cornell. The primary goal of DPW is to empower participants through workshops, campus tours, and familiarization with the application process. 

To increase financial security and reduce the financial burden on the participating departments of DPW, I have been working with several co-leaders to obtain outside funding sources. I put together an infographic (\href{https://github.com/nsantantonio/DPWinfographic/blob/master/infoGraphic.pdf}{link}) to describe how participants self-identify and what the matriculation rates are. This infographic was used to help solicit funding from Corteva, who agreed to sponsor two students to attend DPW in March 2020. %I am currently working to develop a pamphlet to encourage donors to join our effort by sponsoring students to attend DPW. 

Now that a few years of data on participant outcomes has become available, I want to help demonstrate the efficacy of the DPW program and increase the exposure and outreach of DPW. I initiated and then partnered with several DPW co-leaders to submit an abstract for a talk at The Allied Genetics Conference (\href{https://genetics-gsa.org/tagc-2020/}{TAGC}) in April 2020. The talk will be delivered by a graduate student in Entomology, Andrea Darby, to the DEI section of the Ethical, Legal, Social Issues theme at TAGC. Our hope is that this talk can target faculty, staff and students to encourage undergraduates at their own institutions to apply for DPW, or even create similar programs. We also hope to solidify current donors, as well as attract new donors by demonstrating that DPW like programs are effective and that there is interest in such programs in the academic community. I hope to continue support for these types of programs through advocacy as a faculty member.  


\subsection*{The Hidden Curriculum}

% I recently attended the BTI Postgraduate Society (PGS) workshop ``Future Leaders in Plant Science: the Hidden Curriculum'', led by Janani Hariharan (see \href{science.sciencemag.org/content/364/6441/702.full}{Hariharan, Science 2019}). Hariharan demonstrated how social and professional norms to some are not made clear to others, through a role-playing group construction project in which only certain objectives were made clear to some actors. Playing the part of the assistant who is supposed to help without asking questions or knowledge of the project objectives, the exercise had a profound effect on me, shedding light on how frustrating it is to be in the dark when others expect you to conform/perform. \amy{clean up - not clear to the reader, too specific to your experience} \amy {don't like slashes}

% There is a disconnect [and a large degree of uncertainty] \amy{don't like brackets} for many students between what they think is expected of them, and what is actually expected of them. Some students who are more familiar with the US education system learn many untaught concepts that are not necessarily obvious or made accessible to those who are less familiar. For example, many students from underrepresented backgrounds, international students, and \emph{especially} visiting scientists, are given little guidance or resources outside of a desk to sit at. I believe there is almost a reinforcement of this process, where there is an unspoken attitude that ``the best ones will figure it out''. Such a sink or swim policy may identity a few good swimmers who already know how to swim, but it allows future [potential] Olympians to drown before they have ever been taught that they \emph{can} swim. %As a graduate students and postdoc, I have seen this play out many times.

% I want to work to mitigate the 


I recently attended a workshop on the Hidden Curriculum, led by Janani Hariharan (see \href{https://science.sciencemag.org/content/364/6441/702.full}{Hariharan, 2019}). The workshop included a role-playing, group construction project that had a profound effect on me, shedding light on how frustrating it is to be in the dark when others expect you to perform or meet standards unknown to you. 

There is a disconnect for many students between what they think is expected of them, and what is actually expected. Some students who are more familiar with the US education system learn many untaught concepts that are not necessarily obvious or made accessible to those who are less familiar. Students from underrepresented backgrounds, international students, and \emph{especially} visiting scientists, are given little guidance or resources outside of a desk to sit at. I believe there is almost a reinforcement of this process, where the unspoken attitude is ``the best ones will figure it out''. Such a sink or swim policy may identity a few good swimmers, but it also allows future Olympians to drown before they have ever been taught that they \emph{can} swim. %As a 

% Hariharan demonstrated how social and professional norms to some are not made clear to others, through a role-playing group construction project in which only certain objectives were made clear to some actors. Playing the part of the assistant who is supposed to help without asking questions or knowledge of the project objectives, the exercise had a profound effect on me, shedding light on how frustrating it is to be in the dark when others expect you to conform/perform. \amy{clean up - not clear to the reader, too specific to your experience} \amy {don't like slashes}



To help uncover this curriculum, I would like build a requirement for new students and their special committee to produce written expectations and timelines with specific milestones. These milestones could include discrete items such as specific courses, conferences, publications, or exam dates, as well as soft skills such as oral communication, critical thinking, time management and leadership. Expectations for faculty should also be discussed and included in these agreements. This requirement would be a negotiation between the student and the committee that would be updated yearly or semi-yearly, signed by all parties and submitted to the section. This paper trail can then be used to address deficiencies on either side if a conflict arises. 

As faculty, we must take the lead in providing a safe place to ask academic and non-academic questions and strive to make lab members feel comfortable at work. Faculty need training on what norms and expectations they take for granted. This will enable us to clarify and help students access available resources. I intend to include a monthly lab meeting designed to provide a safe place for lab members to learn about one another outside of their research projects, ask questions and discuss the abundant, but underutilized, resources that are available on campus. I would also work with my lab and other section members and faculty to identify and address hidden curriculum topics at this meeting. If this type of meeting proves useful, I will encourage everyone in the section to join in on the discussion, as well as urge other leaders to initiate similar discussions. %to better prepare them for 

% . Faculty members need to help students  they can . How to apply to jobs, negotiate a salary, even how to write a diversity, equity and inclusion statement. 

% , such that  for a negotiation process to define clearly written expectations should be negotiated by the committee and the student not only at onboarding, but on a yearly or semi-yearly basis. 



% a workshop on the hidden curriculum that emphasized the difficulties and inequalities created when  



% I wasn't shown how to meet any of these expectations. I am learning them through trial and, so far, error. I want to incoporate these skills into the written curriculum, by learning what is expected, and . [Amy: dont talk about yourself. experienced by many/most]. 

 % I am continuing to learn how I can contribute, and 

% and addressed ways to mitigate, or uncover this curriculum.

% An issue I have observed as a relatively recent graduate student is the existence of a hidden curriculum. 


\subsection*{Learning DEI concepts in a safe environment}

It is important that we recognize the need for education in DEI concepts and ideas. To enact societal change, individuals need to be willing to be taught and engage in the conversation. During this learning process many will make mistakes, including me. Mistakes are okay! As long as we learn from them and strive to better ourselves and our understanding of and compassion for others. It is paramount that we cultivate an environment where people can learn, practice and ask questions about DEI concepts without fear of retribution if they misstep. 

% Workshops, outside speakers, explicit instruction, inclusive discussion sessions. 



% make mistakes

% I will be continuing to learn about increasing diversity, equity and inclusion at Cornell and in our community. 


 % which would only work to further the divide.  

% environment where people can make mistakes while learning and practicing without fear of retribution being ostricized. help learn how to conversation.



% by explicietly teaching and participating. making people be apart of the . requirements. 

% part of forming sociateal norms that are inclusive such as. 

% For example, accidentally using the wrong pronoun for someone should not turn into a big deal. The right course of action is to simply apologize and make a conscious effort to use the pronouns that the person identifies with in the future. 

% experience of thos oin the  process. Hard to learn and contribute if the language


% \subsection*{Diversity Inclusion and Equity through experience}

% I have taken implicit bias assessments to gauge what my own biases are.

It is also imperative that we, as educators, provide an inclusive environment and work to make resources equally available and expectations clear. We must reevaluate our own practices and implicit biases to ensure we have not unintentionally created an inequitable or exclusive situation. To address this need, I want to work to develop a series of DEI related requirements for SCS faculty, staff and graduate students. 

For faculty and staff, numerous resources for DEI are already available through the Vice President for Diversity office on campus (e.g. \href{https://diversity.colostate.edu/our-programs/#1504017838903-86d3213a-2d2c}{CEIP} and \href{https://diversity.colostate.edu/our-programs/#1504017838952-75b4d525-fe10}{FIIE}). Faculty should be \emph{required} to attend these types of workshops on a regular basis, and a feedback system needs to be promoted to allow for changes to existing workshops, and the creation of new ones based on observed needs. Regular evaluations of how each faculty member works to promote DEI within their own research groups need to be built, so that they can learn how to address their own biases and shortcomings as a leader of a diverse, inclusive and equitable team.
% readily checks in to ensure that individuals have not strayed to far from a path toward success.

I would like to develop a diversity requirement for graduate students with the intent to increase exposure to new people, places and ideas. The requirement would be flexible and up to the student's committee to determine appropriateness. Ideas to meet the requirement could include a community-based project, volunteering to a DEI oriented program such as DPW, or a semester at another university or a CGIAR center. %I also think a diversity of experience in graduate school would also benefit students and faculty. 

To provide such opportunities, I would work to set up a graduate student exchange program between universities in different parts of the country, such as the South and Northeast. A semester in a contrasting environment would expose students to different demographics, cultures, traditions, ideas and farming practices in both directions. Modern jobs in academia and industry have rightly begun to put much more value on DEI, with many expecting direct evidence of a commitment to DEI from the applicant. This requirement would help provide personal experience with DEI ideas and initiatives that will aid students in pursuing jobs that value these important goals.% \nicholas{a contribution to DEI}. 
% Will also help them get jobs later. 

% I developed an infographic to describe the demographics of the participants, and am currently working to develop a pamphlet to encourage donors to join our effort




\subsection*{Lab culture}
 % potential , where I discovered the difficulties they face with consistent funding. The graduate school SIPS has generously donated both funding and staff time for scheduling travel, and learned and 



% heirarchical mentoring -> recent evidence that shows mentorship outside advisor most important for grad students. 
% inclusinve teaching tactics

The most effective teams with the best ideas come from diverse backgrounds and experiences. Women, minorities, LGBTQIA and members of other underrepresented groups continue to face less opportunity and numerous obstacles, especially in STEM. Unfortunately, this has lead to a lack of perceived academic ``merit'' by some, when these differences are purely environmental in nature and driven by legacy and current social biases. I am committed to helping reverse these trends by building a diverse team and an environment in which everyone is valued and given the resources and opportunities they need to succeed. 



% These individuals have also  especially in STEM, and can have lower percieved ``merit'' as a result. fundamentally environmental in nature, driven by current and legacy social biases constructed by rich and powerful white men. 



% https://www.smithsonianmag.com/science-nature/unheralded-women-scientists-finally-getting-their-due-180973082/

% am committed to hiring graduate students, postdocs and staff

% Analytically, ethical hiring decisions can be represented as a selection index problem of perceived merit (both teaching and research) and perceived contribution to DEI. These traits are negatively correlated, as women, minorities and members of other underrepresented groups have historically had less opportunity and numerous obstacles, especially in STEM. Therefore, we must strive to accomplish two goals. First, to properly weight these traits to avoid putting too much emphasis on perceived merit, and second, to break this correlation. This correlation is fundamentally environmental in nature, driven by current and legacy social biases constructed by rich and powerful white men. 

% \subsection*{Lab culture}

A high quality work environment is necessary for high quality work. When undesirable aspects of the work environment cannot be changed (e.g. an office with no windows), high standards and equality in equipment matter even more. Small differences in resources can create an importance hierarchy, even when there is no direct intention to establish one. This includes ensuring that everyone has equally good chairs, desks and monitors. Providing second-hand equipment in poor condition to new lab members suggests they are less important, and is not acceptable.

% high quality equipment (e.g. chairs, standing desks, computer monitors)


%faculty member must also engage with This heirarchy 

Seemingly small differences can be be created inadvertently, and may result in great inequities. For example, holding lab meeting in a room without enough seats at the table. This sets up a stark contrast between those at the table and those seated away from it. New, international and women lab members may not feel welcome to sit at the table, and are therefore largely left out of the conversation. As faculty, we need to identify these situations and strive to find equitable solutions, such as encouraging all members to sit at the table and join in the conversation and finding a room with a table that can accommodate everyone. 


Hierarchical mentoring has been shown to be highly influential on the progress of scientists as they advance through their degrees and postdoctoral work. Lab members with more experience transfer their knowledge and skills to those with less experience, but this does not mean that they are more or less important. As the lab leader, it is critical to foster this process, but also recognize that it cannot replace direct interaction with graduate students. Some students are very comfortable or may even prefer to operate independently, but this is a recipe for disaster if communication between the student and faculty mentor breaks down due to lack of frequency. Both parties are responsible to work to maintain this very important relationship, and should be double checked during yearly reevaluations of expectations.
% DPW recruits people from underrepresented groups from around the United States and US territories that are interested in graduate school. Selected participants are invited to attend a fully-funded weekend at Cornell with the primary goal of empowering participants through workshops, campus tours, and familiarization with the application process. 

% I have actively engaged in learning about diversity, equity and inclusion as well as joined efforts to . 


%heterosexual white males and  % How can I help provide the  and opportunity to give a chance to those less fortunate?

% New Mexico, shocked by the paucity lack of Latin and native americans at Cornell. 


% I recognize the benefit of , as I come from a place with very different demographics.

% homogeneous. 
% While the plant breeding community has seen some progression in the balance of men and women in the field, the same is not necessarily true of different races and ethnicities. Indeed, Plant Breeding at Cornell is rather diverse from a global perspective; however, coming from New Mexico and a Hispanic-Serving Institution (NMSU), I was struck by the relative paucity of \emph{American} minority students. The plant breeding community suffers from a lack of these important cultural perspectives and I would like to help Cornell to be a leader in including these underrepresented groups at a representative level.





% In an effort to address the need for more diversity in STEM graduate programs, graduate students from the Ecology Evolutionary Biology, Entomology and School of Integrative Plant Sciences (SIPS) have joined forces to build the Diversity Preview Weekend (DPW). DPW recruits people from underrepresented groups from around the United States (and US territories) that are interested in graduate school, but have little access to what a graduate degree is like. DPW selects applicant based on merit and need, and pays for them to attend a weekend at Cornell consisting of workshops, tours, familiarization with the applications process and funding sources, as well as networking opportunities. 


% founded the Fundraising Chair for the DPW 2020 cycle. While 

% I joined an initiative to reach out to donors outside the university, in which I developed an infographic to describe the demographics of the participants, and am currently working to develop a pamphlet to encourage donors to join our effort. I have also collaborated with another member of DPW to write an abstract detailing the effectiveness of the program, that we submitted for a talk at The Allied Genetics Conference (TAGC, Genetics Society of America, April 2020).


% Moving forward, I want to establish connections with the ithaca community 


% A relatively new concept to me is the hidden curriculum, where students learn skills/behaviors Fwithout instruction. These cirricula can be a challenge. 

% \section{Lab culture}


% American minorities often have a family history of hardship, and I believe they are an important part of our diverse community that is underrepresented in Plant Breeding at Cornell. 
% For example, there are graduate students from all over the globe, but I could not help but notice the scarcity of \emph{American} minorities.  %experienced or fled extreme hardship in their families past. 


%Notable exceptions include Cornell's Barbara McClintock, who studied maize genetics and transposable elements and who has since become an important figure for women in science. Unfortunately, she was discouraged by her advisor, R. A. Emerson, to pursue graduate degree in plant breeding, supposedly because he felt she would not be eligible for employment with such a degree. 


% Given the mixed history of agriculture and these groups, perhaps it is not surprising that interest in plant breeding is low. %  greater into higher standing? levels? proportions?

% inclusive 

\subsection*{Community outreach}

I intend to help establish community outreach initiatives to identify, engage and recruit underrepresented high school students to the plant sciences. We should target low-income school districts, where young people may be unfamiliar with the career opportunities in agriculture. These careers can provide a way for aspiring young people to make real differences in the lives of themselves and others. %I also intend to build channels with land grant universities in the South and Southwest US to recruit Black, Latino and American Indian undergraduate students to graduate study in SIPS. 

In an effort to reach out to local underrepresented communities, I aim to hire one or two students a year from the local high schools for work during the field season. I intend to seek out students who need additional resources to succeed in college, especially those from underrepresented backgrounds and potential first generation students. This work experience would provide these students with familiarity of a working research environment, and will benefit their application to, and success in college. This practical experience should also aid them in landing undergraduate research assistantships, if they choose to pursue them. Funding for such a program would be obtained through local or national granting agencies.

\subsection*{Curriculum}

The curriculum needs to be reflective of the breadth of diversity, such that students recognize a bit of themselves, their culture or contributions that were made by people they can relate to. All cultures have their own traditions and customs for distributing seed, selection and cultivation. These practices are often assigned as unique gender roles, which often differ depending on culture. Gender roles in plant breeding and agriculture have largely been ignored in the curriculum until very recently. It is important that we expose students to these perspectives, which add rich texture to what is currently a rather monotonic Western viewpoint of seed systems. Recognition of self will further add value and engage students with diverse backgrounds.

To incorporate these ideas into the curriculum, I want to explore ways to teach about plant domestication, maintenance of genetic variation, and trait selection from different cultural and gender oriented perspectives. I want to identify and recruit guest class speakers who are familiar with seed systems outside the US and western European systems, many of whom may already be at Virginia Tech. Female and male speakers from American Indian, African and Asian cultures could be invited to discuss how their own seed and agricultural systems function. This will cultivate classroom discussion on how these seed systems differ from those in the US and what the benefits of these systems are. This will also challenge students to think critically about how we might adapt our systems to further food security and empower women in agriculture around the globe. 

% develop relationships with people familiar with  to use my connections in New Mexico to establish relationships with Puebloan farmers in New Mexico and Arizona to bring a perspective on their seed systems through guest lectures to introductory courses, such as PLBRG2010. 



% Other unique seed systems, such as those in Southeast Asia and Africa, should also be introduced through students and researchers familiar with those seed systems. 

%from those areas that are at or collaborate with Cornell. % Native American seed systems tend to encourage maintenance of genetic diversity and unique traits, yet they are not adapted to the modern US agriculture system. 
% through time and space. 

% As widespread agriculture was an integral part of most people's daily lives up until rather recently, 

%Interesting examples include the redistribution of new world crops to old world cultures, where many of the traditional foods of varying old world cultures include these new world crops. 

% Maize has a rich history of domestication and dissemination throughout the world, and different cultures use maize in many different ways.

% \subsection*{a diversity index}

% Analytically, ethical hiring decisions can be represented as a selection index problem of perceived merit (both teaching and research) and perceived contribution to DEI. These traits are negatively correlated, as women, minorities and members of other underrepresented groups have historically had less opportunity and numerous obstacles, especially in STEM. Therefore, we must strive to accomplish two goals. First, to properly weight these traits to avoid putting too much emphasis on perceived merit, and second, to break this correlation. This correlation is fundamentally environmental in nature, driven by current and legacy social biases constructed by rich and powerful white men. 

% However, this would largely ignore the human aspect of choosing not just an employee, but a colleague. 

% I implore you to carefully consider my research and teaching merits, as well as my contribution to DEI, and how each component should be weighted in your decision. In the year since I last applied to this position, I have obtained extramural funding to pursue my research interests in forages, submitted two first author publications from my postdoc, collaborated with international partners, co-instructed a course in advanced quantitative genetics concepts and pursued learning and engaging in DEI initiatives. 


% Many can meet the latter by coming from a diverse background . The question is how much weight should we put on these different traits?


\subsection*{Conclusion}

The opportunity to purse higher education should be accessible to all people regardless of race, culture, ethnicity, gender, sexuality, ability, disability, religion, nationality, socioeconomic status, beliefs or geographic origin. Educators have a duty to provide curriculum and instruction in an inclusive setting that is reflective of our diverse communities, where all perspectives contribute to a wealth of experience and knowledge. We have a responsibility to continue to learn how to adapt our practices and ideas to better serve the community and to prepare future educators to be leaders in promoting diversity, equity and inclusion throughout their careers. %I hope to contribute to the continued legacy of diversity, equity and inclusion at Cornell University. 

% different perspectives backgrounds experiences add to collaborate . multiple perspectives, , increasingly diverse representation, communities, 


% global community different flavors contribute in a meaningful way . 



% Having recently been a grad studnet acutely aware that faculty may not understand to what degree a hidden cirriculum exists
% Hidden cirriculum workshops for faculty


% inclusive lab culture. 
% team effort, 

% heirarchical mentoring -> recent evidence that shows mentorship outside advisor most important. 
% inclusinve teaching tactics


% actively thinking about and learning DEI - > eg hidden cirriculum 

 
\end{document}
