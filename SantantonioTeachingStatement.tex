\documentclass[11pt]{article}
\usepackage{amsmath}
\usepackage{amssymb}
\usepackage[backend=bibtex, style=authoryear]{biblatex}

\usepackage{xcolor}

\definecolor{hyperblue}{rgb}{0,0,0.7}

\usepackage{hyperref}[hidelinks]

\hypersetup{
	colorlinks = true, 
	urlcolor = hyperblue, 
	linkcolor = hyperblue, 
    anchorcolor = hyperblue,
    citecolor = hyperblue,
    filecolor = hyperblue,
     }

\textwidth=470pt
\oddsidemargin=0pt
\topmargin=0pt
\headheight=0pt
\textheight=650pt
\headsep=0pt

\pdfpagewidth=\paperwidth
\pdfpageheight=\paperheight

\title{Statement of Research}


\author{Nicholas Santantonio}
\date{\today}


\newcommand{\gxe}{G$\times$E}
\newcommand{\gxg}{G$\times$G}


\begin{document}

\section*{\centering Statement of Teaching}
\begin{center} Nicholas Santantonio \end{center}

\noindent Plant breeders have traditionally been generalists, combining genetics with a range of plant sciences to identify farmers' needs and turn out products to meet those needs. However, the range of skills required in the field is rapidly increasing. Proficiency in statistics, programming, bioinformatics and machine learning are now expected in addition to the traditional skills in physiology, pathology, agronomy and genetics. As educators, we must update our curriculum to best prepare students for careers in the new era of genomics and digital agriculture. 

The widening breadth of the plant breeding discipline may demand multiple paths of instruction, which should include a more structured series of quantitative courses to benefit students seeking specialization. Students who chose this path would finish graduate school with the comprehension and ability to effectively adapt the latest computational techniques to meet future breeding goals. I plan to work closely with the new wheat breeding professor to develop quantitative courses that complement the plant breeding course material.

\subsection*{Teaching experience}

During graduate school, I led several activity-based learning sections as a Teaching Assistant (TA) for an introductory biology course at Cornell, \href{https://classes.cornell.edu/browse/roster/SP19/class/BIOMG/1350}{BIOMG1350}. As a TA for an introductory plant breeding course, \href{https://classes.cornell.edu/browse/roster/FA17/class/PLBRG/2010}{PLBRG2010}, I built a pdf based online homework medium, managed the course website, coordinated labs, aided in exam development, and constructed and delivered several lectures. In my third year of graduate school, I was asked by several members of the second year cohort to lead a weekly discussion of a quantitative genetics textbook (\href{http://stemmapress.com/}{Bernardo, 2010}). As a postdoc, I continued to hone my teaching experience and philosophy by co-instructing an advanced graduate-level course on the evolution of genetic modeling in plant breeding, \href{https://classes.cornell.edu/browse/roster/FA19/class/PLBRG/7420}{PLBRG7420}.

\subsection*{Teaching philosophy}

Inclusive teaching tactics are key to ensuring equal access to knowledge in the classroom. Expectations must be made clear, and reinforced throughout the semester so that students do not get behind and fail to meet milestones. It is important for students to be exposed to plant breeding and quantitative genetics ideas from multiple perspectives, and they must be given the opportunity to demonstrate critical thinking on different levels. While some students may show analytical thinking and synthesis on exams, others may shine in more hands-on projects. Most mathematical and computational learning occurs through doing, not through watching. The lecture is important to present material in a concise, structured manner, but concepts are cemented when the student can reconstruct the ideas on their own time.


Longer term projects provide students the opportunity to practice and apply concepts in a more autonomous environment, and are invaluable for assessing comprehension and critical thinking. Regularly assigned homework and hands-on labs are crucial for evaluation of the pace and overall understanding of the material presented. Greenhouse and field facilities on campus provide opportunities to get students out of the classroom to see plant breeding in practice. Public databases provide resources where students can get experience working with real datasets. Computer simulations are useful tools to evaluate comprehension, where in order to simulate a system correctly, the student must understand that system well.

All courses I instruct would contain a term project of relevant complexity to augment exams, in-class labs and homework assignments. For quantitatively oriented courses, these projects would be computational in nature, where students would use real or simulated data to explore the ideas covered in the course. They would then be asked to present their results in written and oral formats that mirror typical scientific communication. Projects may be team oriented to promote collaborative skills and project management. %For courses without a significant quantitative aspect, term projects may be formulated as a research proposal.

\subsection*{Courses}

I intend to develop a yearly, advanced undergraduate course in quantitative genetics to augment the undergraduate plant breeding course. This course would target $21^\text{st}$ century plant breeding concepts with a focus on the use of genome-wide information to drive decision making. In addition, I intend to develop an advanced graduate course that would be held every other year, focusing on the methodology of complex quantitative ideas. Both courses would include a hands-on computational component to reflect the skills currently desired in the field.

The undergraduate course would start with basic probability theory and the single locus model, advancing through genome-wide association, genomic prediction and selection theory. Computational labs would be used to augment student understanding of course material. Students would use available computational tools to analyze small example datasets with a focus on interpretation of results. For the term-project, students will be split into groups, and given a breeding scenario and a dataset. They will determine the genetic architecture of their trait and propose a breeding strategy based on their scenario and what they can learn from the data. By the end of the semester, students will be able to demonstrate critical thinking of plant breeding methods and ideas, with the ability to synthesize when given new plant systems or breeding goals. 

The graduate quantitative genetics course would shift focus onto the methodology of more complex ideas, including coalescent theory, hierarchical Bayesian models, spatial variation, G$\times$E, multivariate selection indices, longitudinal models and optimal contribution. Students will be expected to write their own software to solve computational plant breeding problems presented in weekly computational labs. Assignments would be required to be submitted as typed documents in Markdown, \LaTeX\ or similar format, to expose students to more effective modes of mathematical communication outside of Microsoft Office. The term project would then consist of groups of two to four students finding a genotype-phenotype dataset, and working to develop a genotype to phenotype map, assess genomic predictability and construct an optimized breeding scheme throughout the semester. 

\subsection*{Curriculum}

Currently, most life science students do not acquire in-depth statistics and programming skills until graduate school, impeding their progress while they learn to grapple with these new languages. Moving forward, I would like to work with faculty in SCS, statistics and computer science to build a quantitative/computational genetics undergraduate curriculum. Genetics is a vast field; it is imperative that students get exposure to and training in the rapidly changing environment in which they will soon be seeking jobs. Students in essentially all sub-fields of genetics will need to be able to deal with large datasets, using tailored algorithms to make inferences, predict the unobserved, and guide decision making. Big data management requires skills in programming, linear algebra, statistics and machine learning that must be incorporated into the curriculum at earlier stages.


Genomics and digital agriculture are only just starting to change the landscape of food production. Quantitative skills are one of the specializations imperative in plant breeding, and CSU must be at the forefront of best preparing the individuals who will usher in this new era.

\end{document}
